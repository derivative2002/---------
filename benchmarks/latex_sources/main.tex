\documentclass[12pt,a4paper]{article}
\usepackage{ctex} % 中文支持
\usepackage{geometry}
\usepackage{pgfplots}
\usepackage{tikz}
\usepackage{booktabs}
\usepackage{multirow}
\usepackage{array}
\usepackage{colortbl}
\usepackage{float}
\usepackage{caption}

% 页面设置
\geometry{left=2.5cm,right=2.5cm,top=2.5cm,bottom=2.5cm}

% 图表设置
\pgfplotsset{compat=1.17}
\usetikzlibrary{patterns,shapes.geometric,arrows.meta,positioning}

% 标题设置
\title{星际争霸II五大精英单位评估报告}
\author{数学建模小组}
\date{\today}

\begin{document}
\maketitle

\section{综合评分天梯榜}

\begin{table}[H]
\centering
\caption{五大精英单位综合评分}
\begin{tabular}{cllrrr}
\toprule
排名 & 单位 & 指挥官 & 综合评分 & CEV & 资源效率 \\
\midrule
\rowcolor{yellow!30} 1 & 天罚行者 & 阿拉纳克 & 30.0 & 46.5 & 5.2 \\
\rowcolor{gray!20} 2 & 掠袭解放者 & 诺娃 & 26.5 & 42.1 & 3.1 \\
\rowcolor{orange!20} 3 & 攻城坦克 & 斯旺 & 16.9 & 24.9 & 4.9 \\
4 & 穿刺者 & 德哈卡 & 13.6 & 20.0 & 4.0 \\
5 & 龙骑士 & 阿塔尼斯 & 6.0 & 8.1 & 2.9 \\
\bottomrule
\end{tabular}
\label{tab:scores}
\end{table}

{\small 评分说明:综合评分 = CEV × 60\% + 资源效率 × 40\%}


\section{多维度性能对比}

\subsection{场景排名对比}

\begin{figure}[H]
\centering
\begin{tikzpicture}
\begin{axis}[
    width=12cm,
    height=8cm,
    ybar,
    bar width=0.12cm,
    xlabel={单位},
    ylabel={排名(逆序)},
    symbolic x coords={龙骑士,天罚行者,攻城坦克,穿刺者,掠袭解放者},
    xtick=data,
    ymin=0,ymax=5.5,
    legend pos=north west,
    grid=major,
    grid style={dashed,gray!30},
]

% 这里需要填充实际数据
\addplot coordinates {(龙骑士,3) (天罚行者,5) (攻城坦克,2) (穿刺者,1) (掠袭解放者,4)};
\addplot coordinates {(龙骑士,2) (天罚行者,4) (攻城坦克,3) (穿刺者,1) (掠袭解放者,5)};
\addplot coordinates {(龙骑士,4) (天罚行者,5) (攻城坦克,1) (穿刺者,1) (掠袭解放者,3)};
\addplot coordinates {(龙骑士,3) (天罚行者,5) (攻城坦克,2) (穿刺者,1) (掠袭解放者,4)};
\addplot coordinates {(龙骑士,5) (天罚行者,5) (攻城坦克,4) (穿刺者,3) (掠袭解放者,2)};

\legend{总体,对地,对空,对轻甲,对重甲}
\end{axis}
\end{tikzpicture}
\caption{不同场景下的单位排名对比}
\label{fig:ranking}
\end{figure}


\subsection{性能雷达图}

\begin{figure}[H]
\centering
\begin{tikzpicture}
% 雷达图需要更复杂的TikZ代码
% 这里提供一个简化版本
\node[text width=10cm, align=center] {
    \textit{雷达图展示五个维度的性能对比}\\
    \textit{(总体、对地、对空、对轻甲、对重甲)}\\[1em]
    天罚行者在所有维度都表现优异
};
\end{tikzpicture}
\caption{多维度性能雷达图}
\label{fig:radar}
\end{figure}


\section{排名热力图}

\begin{figure}[H]
\centering
\begin{tikzpicture}
\begin{axis}[
    colormap/RdYlGn,
    colorbar,
    point meta min=1,
    point meta max=5,
    width=10cm,
    height=6cm,
    xtick={0,1,2,3,4},
    xticklabels={总体,对地,对空,对轻甲,对重甲},
    ytick={0,1,2,3,4},
    yticklabels={龙骑士,天罚行者,攻城坦克,穿刺者,掠袭解放者},
]
\addplot[matrix plot*,point meta=explicit] table[meta=C] {
    x y C
    0 0 3
    1 0 4
    2 0 2
    3 0 3
    4 0 1
    % 更多数据...
};
\end{axis}
\end{tikzpicture}
\caption{单位场景排名热力图}
\label{fig:heatmap}
\end{figure}


\section{结论}

根据综合评估,天罚行者在多个维度表现最佳,是五大精英单位中的最强者。
掠袭解放者在对地作战中表现突出,而龙骑士在对重甲单位时具有极高的性价比。

\end{document}
