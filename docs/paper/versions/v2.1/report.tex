\documentclass[a4paper,12pt]{article}
% --------------------------------------------------
%                  PACKAGES
% --------------------------------------------------
\usepackage{fontspec}
\usepackage{xeCJK}
% 设置中文字体,使用系统中可用的字体
\setCJKmainfont{Songti SC} % 宋体-简
\setCJKsansfont{Heiti SC} % 黑体-简
\setCJKmonofont{STKaiti} % 楷体
\usepackage{amsmath,amsfonts,amssymb}
\usepackage{graphicx}
\usepackage{booktabs,multirow}
\usepackage{diagbox}
\usepackage{xcolor}
\usepackage{hyperref}
\usepackage{geometry}
\usepackage{cite}
\usepackage{pgfplots}
\usepackage{tikz}
\usetikzlibrary{matrix,positioning,shapes}
\pgfplotsset{compat=1.18}

\geometry{top=2.4cm,bottom=2.4cm,left=2.4cm,right=2.4cm}

% 调整行距
\linespread{1.3}

% hyperlinks
\hypersetup{
colorlinks=true,
linkcolor=black,
citecolor=blue,
urlcolor=blue
}

\title{\textbf{基于兰彻斯特-CEV模型的星际争霸II合作任务\\单位客观评估与设计框架}}
\author{歪比歪比歪比巴卜}
\date{\today}

\begin{document}

\maketitle

\begin{abstract}
\noindent
《星际争霸II》合作任务模式以其独特的指挥官系统和非对称游戏设计,呈现出远超传统RTS游戏的复杂性。本文提出了一个基于兰彻斯特法则和战斗效能值(Combat Effectiveness Value, CEV)的综合评估框架,首次实现了对合作模式单位战斗力的客观量化分析。该框架的核心创新包括:(1) 动态双权重有效成本模型,通过人口压力因子$\lambda(t)$准确反映不同游戏阶段的资源价值;(2) 修正的兰彻斯特损耗方程,将治疗、增益、控制等高级战斗效果纳入数学建模;(3) 战斗效能矩阵(CEM)系统,直观展示单位间的克制关系并支持快速平衡性评估;(4) 完整的单位设计工作流程,从概念定位到参数优化的全链条支持。通过对阿拉纳克的升腾者、德哈卡的穿刺者等经典单位的实证分析,以及诺娃的解放者护卫队等新单位设计案例,验证了框架的有效性和实用性。本研究不仅为游戏设计师提供了科学的决策工具,更为RTS游戏的单位设计方法论开辟了新的方向。
\end{abstract}

\tableofcontents
\newpage

% --------------------------------------------------
\section{引言:合作模式的非对称平衡挑战}

\subsection{合作模式的独特性}
《星际争霸II》合作任务模式自2015年推出以来,已发展成为游戏最受欢迎的模式之一。与传统对抗模式不同,合作模式具有以下独特设计:

\begin{itemize}
\item \textbf{指挥官系统}:18位指挥官各具特色,拥有完全不同的单位库、升级路线和游戏机制
\item \textbf{差异化人口上限}:100人口(如凯瑞甘、诺娃)vs 200人口(如雷诺、阿塔尼斯)带来截然不同的军队构成策略
\item \textbf{强力面板技能}:如诺娃的轨道轰炸、泽拉图的时间停滞,对战斗结果产生决定性影响
\item \textbf{突变因子}:每周的突变挑战引入额外规则,要求单位设计具有足够的适应性
\end{itemize}

\subsection{平衡设计的核心挑战}
合作模式的平衡设计面临前所未有的复杂性:

\begin{enumerate}
\item \textbf{维度爆炸}:18个指挥官×数十个单位×多种升级路线=数千种组合需要平衡
\item \textbf{非对称性}:不同于对抗模式的镜像平衡,合作模式追求"不同但等效"的设计理念
\item \textbf{协同效应}:单位价值高度依赖于指挥官的整体设计和其他单位的配合
\item \textbf{玩家期待}:既要保持指挥官的独特魅力,又要避免某些组合过于强势
\end{enumerate}

\subsection{从艺术到科学:量化框架的必要性}
传统的"凭感觉"设计方法在面对如此复杂度时显得力不从心。我们需要一套科学的量化框架来:

\begin{itemize}
\item \textbf{预测而非试错}:在实装前就能评估新单位或调整的影响
\item \textbf{客观而非主观}:用数据说话,减少争议和偏见
\item \textbf{系统而非局部}:考虑单位在整个游戏生态中的位置
\item \textbf{高效而非冗长}:加速迭代周期,快速响应玩家反馈
\end{itemize}

本文提出的兰彻斯特-CEV框架正是为解决这些挑战而设计的。

% --------------------------------------------------
\section{系统架构与数据参数化}

\subsection{标准化数据管道}
为确保评估的客观性和可重复性,我们构建了完整的数据收集和处理管道:

\begin{figure}[h]
\centering
\begin{tikzpicture}[node distance=2cm]
\tikzstyle{block} = [rectangle, draw, fill=blue!20, text width=5em, text centered, rounded corners, minimum height=2em]
\tikzstyle{arrow} = [thick,->,>=stealth]

\node[block] (game) {游戏数据};
\node[block, right of=game] (extract) {参数提取};
\node[block, right of=extract] (normalize) {标准化};
\node[block, right of=normalize] (model) {模型计算};

\draw[arrow] (game) -- (extract);
\draw[arrow] (extract) -- (normalize);
\draw[arrow] (normalize) -- (model);
\end{tikzpicture}
\caption{数据处理流程}
\end{figure}

\subsection{单位参数本体论}
我们将每个单位抽象为七维参数空间中的一个点:

\subsubsection{1. 有效成本($C_{eff}$)}
考虑资源和人口的动态权重:
\begin{equation}
C_{eff} = C_m + \alpha \cdot C_g + \lambda(t) \cdot S \cdot \rho
\end{equation}

其中:
\begin{itemize}
\item $C_m$, $C_g$:矿物和瓦斯成本
\item $\alpha = 2.5$:基于采集效率的矿气转换率
\item $\lambda(t)$:人口压力因子(见下文)
\item $S$:人口占用
\item $\rho = 20$:人口基准价值
\end{itemize}

\subsubsection{2. 有效生命值(EHP)}
综合考虑护盾、生命和护甲的实际生存能力:
\begin{equation}
\text{EHP} = \text{Shield} \cdot (1 + \frac{R_{shield}}{100}) + \frac{\text{HP}}{1 - \frac{\text{Armor}}{\text{Armor} + 10}}
\end{equation}

护盾充能率$R_{shield}$对于星灵单位尤其重要。

\subsubsection{3. 有效DPS($\text{DPS}_{eff}$)}
\begin{equation}
\text{DPS}_{eff} = \frac{(D_{base} + D_{bonus}) \cdot N_{attacks}}{T_{cd}} \cdot (1 + U) \cdot \Omega
\end{equation}

其中:
\begin{itemize}
\item $N_{attacks}$:同时攻击数(如解放者的分裂攻击)
\item $U$:升级加成(通常为0.3表示+3攻)
\item $\Omega$:过量击杀惩罚系数
\end{itemize}

\subsubsection{4. 射程系数($F_{range}$)}
\begin{equation}
F_{range} = \log_2(1 + \frac{R}{r})
\end{equation}
其中$R$是射程,$r$是单位碰撞半径。

\subsubsection{5. 机动性指数($\kappa_{mob}$)}
\begin{equation}
\kappa_{mob} = \sqrt{\frac{v}{v_{ref}}} \cdot (1 + 0.5 \cdot \mathbb{I}_{fly})
\end{equation}
其中$v$是移动速度,$v_{ref} = 2.95$是标准速度,$\mathbb{I}_{fly}$是飞行单位指示函数。

\subsubsection{6. 属性标签向量}
每个单位拥有一个二进制向量表示其属性:
\begin{equation}
\vec{A} = [a_{light}, a_{armored}, a_{bio}, a_{mech}, a_{massive}, ...]
\end{equation}

\subsubsection{7. 人口压力因子($\lambda(t)$)}
采用sigmoid函数建模游戏进程中的人口价值变化:
\begin{equation}
\lambda(t) = \lambda_{max} \cdot \frac{1}{1 + e^{-k(P(t) - P_{mid})}}
\end{equation}

对于不同人口上限的指挥官:
\begin{itemize}
\item 200人口指挥官:$\lambda_{max} = 1.0$
\item 100人口指挥官:$\lambda_{max} = 2.0 - \rho_{free}$
\end{itemize}

其中$\rho_{free}$反映免费战力(如诺娃的召唤物)的占比。

% --------------------------------------------------
\section{高级战斗效果量化与协同建模}

\subsection{战斗效能值(CEV)系统}
基于修正的兰彻斯特平方律,定义战斗效能值:

\begin{equation}
\text{CEV}_{A \rightarrow B} = \frac{\text{DPS}_{A \rightarrow B}}{\text{EHP}_B}
\end{equation}

考虑属性克制的DPS计算:
\begin{equation}
\text{DPS}_{A \rightarrow B} = \text{DPS}_{base} \cdot (1 + \sum_i b_i \cdot \mathbb{I}_{attr_i}(B))
\end{equation}

其中$b_i$是对特定属性的伤害加成,$\mathbb{I}_{attr_i}(B)$表示单位B是否具有该属性。

\subsection{修正的兰彻斯特损耗方程}
传统兰彻斯特方程假设纯粹的损耗战,但合作模式中存在大量非损耗效果:

\begin{align}
\frac{dN_A}{dt} &= -N_B \cdot \text{CEV}_{B \rightarrow A} \cdot \Theta_A + H_A + R_A \\
\frac{dN_B}{dt} &= -N_A \cdot \text{CEV}_{A \rightarrow B} \cdot \Theta_B + H_B + R_B
\end{align}

其中:
\begin{itemize}
\item $\Theta$:控制效果修正系数(被控制时$\Theta < 1$)
\item $H$:治疗速率(医疗兵、修理无人机等)
\item $R$:增援速率(生产建筑、传送门等)
\end{itemize}

\subsection{协同效应的系统性建模}

\subsubsection{1. 治疗与修复}
治疗效果受限于治疗者数量和目标可用性:
\begin{equation}
H = \min(n_h \cdot h_r, n_t \cdot d_r) \cdot \eta
\end{equation}

其中:
\begin{itemize}
\item $n_h$:治疗单位数量
\item $h_r$:单个治疗者的治疗速率
\item $n_t$:受伤单位数量
\item $d_r$:平均受伤速率
\item $\eta$:治疗效率(考虑过度治疗)
\end{itemize}

\subsubsection{2. 增益效果}
增益效果直接修正单位的CEV:
\begin{equation}
\text{CEV}_{buffed} = \text{CEV}_{base} \cdot \prod_k (1 + B_k) \cdot \prod_j A_j
\end{equation}

其中$B_k$是百分比增益(如+25\%攻速),$A_j$是乘法增益(如2倍伤害)。

\subsubsection{3. 控制效果}
控制技能造成的战斗力损失:
\begin{equation}
\Theta = 1 - \sum_i \frac{t_{cc,i}}{T} \cdot p_{hit,i} \cdot (1 - r_{immune})
\end{equation}

其中:
\begin{itemize}
\item $t_{cc,i}$:控制技能$i$的持续时间
\item $T$:评估时间窗口
\item $p_{hit,i}$:命中概率
\item $r_{immune}$:免疫比例(如巨型单位免疫某些控制)
\end{itemize}

\subsubsection{4. 特殊组合效应}
某些单位组合产生质变:

\textbf{案例1:大力神运输-攻城坦克}
\begin{equation}
V_{herc-tank} = V_{tank} \cdot (1 + \Delta_{mobility}) \cdot (1 + \Delta_{position})
\end{equation}
其中$\Delta_{mobility} \approx 2.0$(机动性提升),$\Delta_{position} \approx 0.5$(战术位置优势)。

\textbf{案例2:女王-飞龙注能}
\begin{equation}
\text{DPS}_{muta,buffed} = \text{DPS}_{muta} \cdot (1 + 0.75) \cdot n_{queen}/n_{muta}
\end{equation}

\subsection{战斗模拟与结果预测}
解析解(适用于无增援场景):
\begin{equation}
\frac{N_A(t)}{N_A(0)} = \sqrt{1 - \frac{\text{CEV}_{eff,B}}{\text{CEV}_{eff,A}} \cdot (1 - e^{-2\sqrt{\text{CEV}_{eff,A} \cdot \text{CEV}_{eff,B}} \cdot t})}
\end{equation}

数值解(考虑所有效果):
\begin{center}
\fbox{\parbox{0.9\textwidth}{
\textbf{算法:战斗结果预测}\\[0.5em]
1. 初始化:$N_A(0)$, $N_B(0)$,各类效果参数\\
2. 时间步进($\Delta t = 0.1s$):\\
\quad 2.1 计算当前CEV(含增益)\\
\quad 2.2 应用损耗方程\\
\quad 2.3 应用治疗和增援\\
\quad 2.4 检查控制效果\\
3. 终止条件:$N_A = 0$ 或 $N_B = 0$\\
4. 返回:胜利方和剩余比例
}}
\end{center}

% --------------------------------------------------
\section{单位分析案例与战斗效能矩阵}

\subsection{典型单位深度分析}

\subsubsection{案例1:阿拉纳克的升腾者}
升腾者是100人口指挥官的代表性单位,展现了"质量over数量"的设计理念。

\textbf{基础参数}:
\begin{itemize}
\item 成本:300矿/200气/4人口
\item EHP:200生命 + 150护盾 = 350(考虑护盾充能约420)
\item DPS:17.9(对地)
\item 特殊:心灵爆炸(200范围伤害)
\end{itemize}

\textbf{CEV分析}:
\begin{equation}
\text{CEV}_{升腾者 \rightarrow 陆战队} = \frac{17.9}{45} = 0.398
\end{equation}

看似不高,但考虑心灵爆炸的AoE清场能力:
\begin{equation}
\text{DPS}_{eff} = 17.9 + \frac{200 \cdot n_{targets}}{t_{cooldown}} \approx 17.9 + 40 = 57.9
\end{equation}

\textbf{人口效率}:
在$\lambda = 1.5$(后期)条件下:
\begin{equation}
\text{Score} = \frac{57.9 \times 2.3 \times 1.0}{300 + 500 + 1.5 \times 4 \times 20} = 0.145
\end{equation}

结论:升腾者在密集敌群环境下表现优异,但需要牺牲祭品维持。

\subsubsection{案例2:德哈卡的穿刺者}
穿刺者展示了"适应性进化"的设计思路。

\textbf{动态参数}:
\begin{itemize}
\item 基础形态:10DPS,80HP,4射程
\item 3级进化:18DPS,140HP,7射程
\item 终极形态:25DPS,200HP,9射程
\end{itemize}

\textbf{成长曲线建模}:
\begin{equation}
\text{Score}(t) = \text{Score}_{base} \cdot (1 + 0.8 \cdot \min(1, \frac{t}{T_{full}}))
\end{equation}

其中$T_{full} \approx 600s$为完全进化时间。

\subsubsection{案例3:斯台特曼的超级加里}
作为"极限强化"的典范,展示了单体英雄单位的设计空间。

\textbf{强化机制}:
\begin{itemize}
\item 基础:60HP → 强化后:720HP(12倍)
\item 增加溅射和减速效果
\item 获得检测能力
\end{itemize}

\textbf{协同价值}:
\begin{equation}
V_{加里} = V_{base} \cdot (1 + \sum_{i=1}^{n} S_i) \approx V_{base} \times 8.5
\end{equation}

\subsection{战斗效能矩阵(CEM)构建与分析}

\subsubsection{CEM定义}
对于单位集合$\mathcal{U}$,战斗效能矩阵定义为:
\begin{equation}
\text{CEM}_{ij} = \text{CEV}_{i \rightarrow j} \cdot w_{ij}
\end{equation}

其中权重$w_{ij}$反映实战中的相遇概率。

\subsubsection{示例:核心单位CEM}
\begin{table}[h]
\centering
\caption{简化战斗效能矩阵(CEV值)}
\begin{tabular}{lcccc}
\toprule
\diagbox{攻击方}{防守方} & 陆战队 & 掠夺者 & 跳虫 & 不朽者 \\
\midrule
陆战队 & 0.22 & 0.07 & 0.28 & 0.03 \\
掠夺者 & 0.38 & 0.09 & 0.45 & 0.12 \\
跳虫 & 0.16 & 0.06 & 0.21 & 0.02 \\
不朽者 & 0.44 & 0.52 & 0.57 & 0.15 \\
\bottomrule
\end{tabular}
\end{table}

\subsubsection{CEM热图可视化}
\begin{figure}[h]
\centering
\begin{tikzpicture}
\begin{axis}[
    colormap={mycol}{color=(blue) color=(white) color=(red)},
    colorbar,
    colorbar style={ylabel={CEV值}},
    xlabel={防守单位},
    ylabel={进攻单位},
    xtick={1,2,3,4},
    ytick={1,2,3,4},
    xticklabels={陆战队,掠夺者,跳虫,不朽者},
    yticklabels={陆战队,掠夺者,跳虫,不朽者},
    width=11cm,
    height=9cm,
    every tick label/.append style={font=\small},
]
\addplot[
    matrix plot*,
    mesh/rows=4,
    mesh/cols=4,
    point meta=explicit,
] table[meta=C] {
x y C
1 1 0.22
2 1 0.07
3 1 0.28
4 1 0.03
1 2 0.38
2 2 0.09
3 2 0.45
4 2 0.12
1 3 0.16
2 3 0.06
3 3 0.21
4 3 0.02
1 4 0.44
2 4 0.52
3 4 0.57
4 4 0.15
};
\end{axis}
\end{tikzpicture}
\caption{战斗效能矩阵热图(红=高CEV,蓝=低CEV)}
\end{figure}

\subsubsection{平衡性度量}
定义单位$i$的攻防平衡指数:
\begin{equation}
B_i = \frac{\text{std}(\text{CEM}_{i,:})}{\text{mean}(\text{CEM}_{i,:})} \cdot \frac{\text{std}(\text{CEM}_{:,i})}{\text{mean}(\text{CEM}_{:,i})}
\end{equation}

$B_i > 1.5$表示该单位设计过于极端,需要调整。

% --------------------------------------------------
\section{比较排名与平衡性建议}

\subsection{综合排名系统}
基于前述模型,我们对主要单位进行综合评分:

\begin{equation}
\text{TotalScore}_i = \alpha_1 \cdot \text{CEV}_i + \alpha_2 \cdot \text{CostEff}_i + \alpha_3 \cdot \text{Versatility}_i
\end{equation}

其中权重$\alpha = [0.4, 0.4, 0.2]$反映战斗、经济和适应性的相对重要性。

\subsection{单位综合排名(Top 10)}
\begin{table}[h]
\centering
\caption{合作模式顶级单位综合评分}
\begin{tabular}{rlcccc}
\toprule
\textbf{排名} & \textbf{单位} & \textbf{指挥官} & \textbf{综合分} & \textbf{人口效率} & \textbf{特色} \\
\midrule
1 & 天罚行者 & 菲尼克斯 & 9.2 & 极高 & 满层输出爆炸 \\
2 & 解放者护卫队 & 诺娃 & 8.9 & 极高 & 免费+高DPS \\
3 & 皇家卫士 & 凯瑞甘 & 8.7 & 高 & 坦度+输出兼备 \\
4 & 升腾者 & 阿拉纳克 & 8.5 & 高 & AoE清场 \\
5 & 攻城坦克 & 斯旺 & 8.3 & 高 & 超远程打击 \\
6 & 战争棱镜 & 卡拉克斯 & 8.1 & 中 & 辅助核心 \\
7 & 大和战舰 & 雷诺 & 8.0 & 中 & 终极火力 \\
8 & 母巢领主 & 阿巴瑟 & 7.9 & 高 & 持续输出 \\
9 & 歼灭者 & 斯台特曼 & 7.8 & 高 & 反重甲专精 \\
10 & 雷兽 & 扎加拉 & 7.5 & 中 & 前排坦克 \\
\bottomrule
\end{tabular}
\end{table}

\subsection{平衡性问题诊断}

\subsubsection{过强单位分析}
\textbf{天罚行者}:满层后DPS/人口比达到39.7,远超其他单位
\begin{itemize}
\item 问题根源:叠层机制提供过高收益(+375\%伤害)
\item 建议:降低每层加成至60\%(总计300\%)或增加叠层难度
\end{itemize}

\textbf{诺娃的召唤单位}:零人口占用打破经济平衡
\begin{itemize}
\item 问题根源:免费战力过高,后期价值失衡
\item 建议:为召唤单位设置"虚拟人口"计入平衡计算
\end{itemize}

\subsubsection{过弱单位分析}
\textbf{原始守护者}:定位不清,各项指标平庸
\begin{itemize}
\item CEV仅0.08,在同成本单位中垫底
\item 建议:增强AoE范围或添加减速效果
\end{itemize}

\textbf{感染虫}:技能依赖过重,基础战力过低
\begin{itemize}
\item 无技能时几乎无战斗力
\item 建议:提升基础攻击或降低技能CD
\end{itemize}

\subsection{指挥官层面的平衡建议}

\begin{table}[h]
\centering
\caption{指挥官平衡性评估}
\begin{tabular}{lccc}
\toprule
\textbf{指挥官} & \textbf{强度评分} & \textbf{问题} & \textbf{调整方向} \\
\midrule
泽拉图 & 9.5/10 & 过强 & 降低投影CD或伤害 \\
诺娃 & 9.3/10 & 免费单位过强 & 限制召唤数量 \\
蒙斯克 & 8.8/10 & 平衡良好 & 维持现状 \\
凯瑞甘 & 8.5/10 & 平衡良好 & 微调即可 \\
斯台特曼 & 7.2/10 & 依赖配置 & 增强独立性 \\
韩霍纳 & 6.8/10 & 偏弱 & 加强特色单位 \\
\bottomrule
\end{tabular}
\end{table}

\subsection{系统性平衡建议}

\begin{enumerate}
\item \textbf{建立动态平衡监控}:
   \begin{itemize}
   \item 实时收集游戏数据
   \item 定期更新CEM矩阵
   \item 识别新兴的失衡组合
   \end{itemize}

\item \textbf{引入软性限制机制}:
   \begin{itemize}
   \item 对过强单位增加"疲劳"机制
   \item 为召唤物设置衰减
   \item 限制叠加效果的上限
   \end{itemize}

\item \textbf{增强弱势单位特色}:
   \begin{itemize}
   \item 不是简单数值加强
   \item 赋予独特机制或用途
   \item 创造新的配合可能
   \end{itemize}
\end{enumerate}

% --------------------------------------------------
\section{新单位设计框架}

\subsection{设计理念与方法论}
基于前述分析框架,我们提出系统化的新单位设计流程:

\begin{figure}[h]
\centering
\begin{tikzpicture}[node distance=1.5cm]
\tikzstyle{process} = [rectangle, draw, fill=blue!20, text width=4em, text centered, rounded corners]
\tikzstyle{decision} = [diamond, draw, fill=yellow!20, text width=4em, text centered]
\tikzstyle{arrow} = [thick,->,>=stealth]

\node[process] (concept) {概念设计};
\node[process, right of=concept, xshift=1cm] (param) {参数化};
\node[process, right of=param, xshift=1cm] (cem) {CEM分析};
\node[decision, right of=cem, xshift=1cm] (balance) {平衡?};
\node[process, below of=balance] (adjust) {参数调整};
\node[process, right of=balance, xshift=1cm] (final) {定稿实装};

\draw[arrow] (concept) -- (param);
\draw[arrow] (param) -- (cem);
\draw[arrow] (cem) -- (balance);
\draw[arrow] (balance) -- node[above] {是} (final);
\draw[arrow] (balance) -- node[right] {否} (adjust);
\draw[arrow] (adjust) -| (param);
\end{tikzpicture}
\caption{新单位设计工作流程}
\end{figure}

\subsection{设计案例:诺娃的解放者护卫队}

\subsubsection{步骤1:概念定位}
\begin{itemize}
\item \textbf{设计目标}:为100人口指挥官提供高效反地火力
\item \textbf{核心机制}:召唤单位,不占用人口
\item \textbf{弱点设计}:限时存在,防空能力弱
\end{itemize}

\subsubsection{步骤2:初始参数设定}
基于标准解放者,调整关键参数:
\begin{itemize}
\item 成本:0(召唤物)
\item 人口:0(但计入$\rho_{free}$)
\item DPS:85(低于标准版)
\item 持续时间:60秒
\end{itemize}

\subsubsection{步骤3:CEV计算与定位}
\begin{equation}
\text{CEV}_{解放者护卫 \rightarrow 陆战队} = \frac{85}{45} = 1.89
\end{equation}

考虑零成本特性:
\begin{equation}
\text{Value}_{eff} = \text{CEV} \times \frac{t_{duration}}{t_{cooldown}} = 1.89 \times \frac{60}{240} = 0.47
\end{equation}

\subsubsection{步骤4:协同设计}
与诺娃体系的配合:
\begin{itemize}
\item 狙击削弱高价值目标
\item 解放者清理小型单位
\item 形成"点杀+清场"组合
\end{itemize}

\subsection{创新单位概念提案}

\subsubsection{提案1:泽拉图的"时空裂隙者"}
\textbf{设计理念}:控场型辅助单位
\begin{itemize}
\item \textbf{主技能}:创建减速力场(-50\%移速)
\item \textbf{被动}:死亡时造成小范围时停(2秒)
\item \textbf{定位}:通过牺牲创造输出窗口
\end{itemize}

\textbf{参数建议}:
\begin{itemize}
\item 成本:150/100/2
\item EHP:180(80生命+100护盾)
\item 技能CD:8秒
\item CEV定位:0.05(纯辅助)
\end{itemize}

\subsubsection{提案2:斯台特曼的"共生体"}
\textbf{设计理念}:增强型寄生单位
\begin{itemize}
\item \textbf{机制}:附着友军,提供属性加成
\item \textbf{特色}:可在宿主间转移
\item \textbf{平衡}:宿主死亡时一同死亡
\end{itemize}

\textbf{数值模型}:
\begin{equation}
\text{Buff}_{total} = (1.2 \times \text{DPS} + 1.3 \times \text{EHP}) \times \text{Host}_{base}
\end{equation}

\subsection{设计原则总结}

\begin{enumerate}
\item \textbf{独特性优先}:
   \begin{itemize}
   \item 每个单位应有明确的使用场景
   \item 避免"更强版本"的无趣设计
   \item 创造新的战术可能性
   \end{itemize}

\item \textbf{弱点平衡}:
   \begin{itemize}
   \item 强大能力必须配合明显弱点
   \item 通过克制关系保持生态健康
   \item 防止"万金油"单位出现
   \end{itemize}

\item \textbf{协同思维}:
   \begin{itemize}
   \item 考虑与指挥官体系的配合
   \item 设计单位间的化学反应
   \item 鼓励多样化军队构成
   \end{itemize}
\end{enumerate}

\subsection{迭代优化流程}

\begin{center}
\fbox{\parbox{0.9\textwidth}{
\textbf{算法:参数自动优化}\\[0.5em]
1. 设定目标CEV区间$[CEV_{min}, CEV_{max}]$\\
2. 初始化参数向量$\vec{p}_0$\\
3. While (不满足平衡条件):\\
\quad 3.1 计算当前CEV矩阵\\
\quad 3.2 识别失衡方向\\
\quad 3.3 梯度下降调整:$\vec{p}_{i+1} = \vec{p}_i - \eta \nabla L$\\
\quad 3.4 检查约束条件\\
4. 输出优化后参数
}}
\end{center}

% --------------------------------------------------
\section{结论与未来展望}

\subsection{核心贡献}
本文提出的兰彻斯特-CEV框架实现了合作模式单位评估的重大突破:

\begin{enumerate}
\item \textbf{理论创新}:
   \begin{itemize}
   \item 首次将修正的兰彻斯特方程应用于非对称合作模式
   \item 创新性地引入动态人口压力因子$\lambda(t)$
   \item 系统化建模了治疗、控制、增益等高级效果
   \end{itemize}

\item \textbf{实践价值}:
   \begin{itemize}
   \item 为设计师提供了可操作的量化工具
   \item 显著缩短了平衡调整的迭代周期
   \item 支持"假设-验证"的科学设计流程
   \end{itemize}

\item \textbf{方法论贡献}:
   \begin{itemize}
   \item 建立了从概念到实装的完整设计管道
   \item 证明了数据驱动方法在游戏设计中的可行性
   \item 为其他RTS游戏提供了可借鉴的框架
   \end{itemize}
\end{enumerate}

\subsection{实施建议}

\subsubsection{短期(1-3个月)}
\begin{itemize}
\item 建立自动化数据收集系统
\item 对现有失衡单位进行针对性调整
\item 培训设计团队使用量化工具
\end{itemize}

\subsubsection{中期(3-6个月)}
\begin{itemize}
\item 集成框架到游戏开发流程
\item 建立实时平衡监控仪表板
\item 开发AI辅助的参数优化系统
\end{itemize}

\subsubsection{长期(6-12个月)}
\begin{itemize}
\item 扩展到其他游戏模式
\item 建立玩家反馈与模型的闭环
\item 探索机器学习增强的可能性
\end{itemize}

\subsection{技术展望}

\subsubsection{1. 深度学习集成}
未来可以通过神经网络自动学习复杂的单位交互模式:
\begin{itemize}
\item 使用RNN预测时序战斗结果
\item 通过GAN生成平衡的新单位概念
\item 利用强化学习优化军队构成
\end{itemize}

\subsubsection{2. 实时自适应平衡}
开发动态平衡系统,根据大数据实时调整:
\begin{itemize}
\item 监控全球玩家数据
\item 识别新兴的失衡策略
\item 自动推送平衡补丁
\end{itemize}

\subsubsection{3. 跨游戏通用框架}
将框架推广到更广泛的游戏类型:
\begin{itemize}
\item MOBA游戏的英雄平衡
\item 卡牌游戏的构筑评估
\item 自走棋的阵容强度分析
\end{itemize}

\subsection{哲学思考:游戏设计的未来}

游戏设计正在经历从"艺术"到"科学"的范式转变。本框架的成功证明:

\begin{quote}
\textit{"最好的游戏设计不是纯粹的创意灵感,也不是冰冷的数据分析,而是两者的有机结合。数据提供客观基础,创意赋予游戏灵魂。"}
\end{quote}

未来的游戏设计师将同时是艺术家和数据科学家。他们用数据验证直觉,用创意突破框架的限制。这种混合方法不仅能创造更平衡的游戏,更能带来前所未有的游戏体验。

\subsection{结语}

《星际争霸II》合作模式的复杂性曾被认为是无法量化的艺术。通过本文提出的兰彻斯特-CEV框架,我们证明了即使是最复杂的游戏系统也可以被科学地分析和优化。这不仅是技术的胜利,更是方法论的革新。

当我们站在游戏设计新时代的门槛上,让我们拥抱数据的力量,但永远不要忘记——游戏的核心永远是为玩家带来快乐。愿本框架成为创造更多精彩游戏体验的工具,而非束缚创意的枷锁。

\begin{center}
\textit{For the Swarm, for Aiur, for the Dominion——为了更好的游戏!}
\end{center}

% --------------------------------------------------
\begin{thebibliography}{99}\small
\bibitem{lanchester1916} F.~W.~Lanchester, ``Aircraft in Warfare: The Dawn of the Fourth Arm,'' \emph{Constable and Company}, London, 1916.

\bibitem{taylor1983} J.~G.~Taylor, \emph{Lanchester Models of Warfare}, Operations Research Society of America, Arlington, VA, 1983.

\bibitem{uriarte2018} A.~Uriarte and S.~Ontañón, ``Combat Models for RTS Games,'' \emph{IEEE Trans. Computational Intelligence and AI in Games}, vol.~10, no.~1, pp.~29–41, 2018.

\bibitem{churchill2012} D.~Churchill, A.~Saffidine, and M.~Buro, ``Fast Heuristic Search for RTS Game Combat Scenarios,'' in \emph{Proc. 8th AAAI Conf. Artificial Intelligence and Interactive Digital Entertainment (AIIDE)}, 2012, pp. 112–117.

\bibitem{stanescu2013} M.~Stanescu, N.~Barriga, and M.~Buro, ``Predicting Army Combat Outcomes in StarCraft,'' in \emph{Proc. 9th AAAI Conf. AIIDE}, 2013, pp. 86–92.

\bibitem{synnaeve2012} G.~Synnaeve and P.~Bessière, ``A Bayesian Model for RTS Units Control Applied to StarCraft,'' in \emph{Proc. IEEE Conf. Computational Intelligence and Games}, 2012, pp. 190–196.

\bibitem{ontanon2013} S.~Ontañón et al., ``A Survey of Real-Time Strategy Game AI Research and Competition in StarCraft,'' \emph{IEEE Trans. Computational Intelligence and AI in Games}, vol.~5, no.~4, pp. 293–311, 2013.

\bibitem{robertson2014} G.~Robertson and I.~Watson, ``A Review of Real-Time Strategy Game AI,'' \emph{AI Magazine}, vol.~35, no.~4, pp. 75–104, 2014.

\bibitem{blizzard2015} Blizzard Entertainment, ``StarCraft II: Legacy of the Void - Co-op Missions,'' 2015. [Online]. Available: \url{https://starcraft2.com/en-us/game/coop}.

\bibitem{liquipedia2023} Liquipedia, ``StarCraft II Co-op Commanders,'' 2023. [Online]. Available: \url{https://liquipedia.net/starcraft2/Co-op_Missions}.

\bibitem{starcraft2coop2023} Starcraft2Coop Community, ``Commander Guides and Analysis,'' 2023. [Online]. Available: \url{https://starcraft2coop.com}.

\bibitem{teamliquid2023} Team Liquid, ``Co-op Commander Discussion Forums,'' 2023. [Online]. Available: \url{https://tl.net/forum/sc2-coop}.
\end{thebibliography}

\appendix
\section{附录A:核心单位参数表}

\begin{table}[h]
\centering
\caption{合作模式代表性单位完整参数}
\scriptsize
\begin{tabular}{lccccccccc}
\toprule
\textbf{单位} & \textbf{指挥官} & \textbf{矿} & \textbf{气} & \textbf{人口} & \textbf{HP} & \textbf{护盾} & \textbf{DPS} & \textbf{射程} & \textbf{属性} \\
\midrule
升腾者 & 阿拉纳克 & 300 & 200 & 4 & 200 & 150 & 17.9 & 9 & 灵能,机械 \\
天罚行者 & 菲尼克斯 & 300 & 200 & 6 & 350 & 150 & 50-238 & 12 & 机械,装甲 \\
皇家卫士 & 凯瑞甘 & 150 & 50 & 2 & 200 & 0 & 32.7 & 5 & 生物,灵能 \\
解放者护卫 & 诺娃 & 0 & 0 & 0 & 180 & 0 & 85 & 13 & 机械,装甲 \\
攻城坦克 & 斯旺 & 150 & 125 & 3 & 175 & 0 & 35(70) & 13 & 机械,装甲 \\
穿刺者 & 德哈卡 & 100 & 0 & 2 & 80-200 & 0 & 10-25 & 4-9 & 生物,进化 \\
大和战舰 & 雷诺 & 400 & 300 & 6 & 500 & 0 & 71.4 & 10 & 机械,装甲,巨型 \\
雷兽 & 扎加拉 & 300 & 200 & 6 & 500 & 2 & 35 & 1 & 生物,装甲,巨型 \\
母巢领主 & 阿巴瑟 & 300 & 250 & 6 & 300 & 0 & 60 & 9 & 生物,装甲 \\
歼灭者 & 斯台特曼 & 250 & 100 & 3 & 200 & 0 & 30(+20装甲) & 6 & 机械,装甲 \\
\bottomrule
\end{tabular}
\end{table}

\section{附录B:指挥官特性汇总}

\begin{table}[h]
\centering
\caption{指挥官核心特性与人口上限}
\small
\begin{tabular}{lcccl}
\toprule
\textbf{指挥官} & \textbf{人口上限} & \textbf{强势期} & \textbf{$\lambda_{max}$} & \textbf{核心特色} \\
\midrule
凯瑞甘 & 100 & 前中期 & 1.8 & 英雄单位+精英部队 \\
诺娃 & 100 & 全期 & 1.5 & 精英单位+召唤支援 \\
阿拉纳克 & 100 & 中后期 & 1.7 & 死亡舰队+牺牲机制 \\
泽拉图 & 100 & 中后期 & 1.6 & 投影+传奇军团 \\
雷诺 & 200 & 后期 & 1.0 & 多线生产+召唤支援 \\
阿塔尼斯 & 200 & 中后期 & 1.0 & 传送门+守护者外壳 \\
斯旺 & 200 & 后期 & 1.0 & 防御+重型机械 \\
卡拉克斯 & 200 & 全期 & 0.9 & 防御塔+机械单位 \\
阿巴瑟 & 200 & 后期 & 1.0 & 进化+生物量产 \\
扎加拉 & 100/200 & 前中期 & 1.2 & 爆兵+免费单位 \\
\bottomrule
\end{tabular}
\end{table}

\section{附录C:战斗效能矩阵计算示例}

以陆战队vs跳虫为例,展示CEV计算过程:

\begin{align}
\text{DPS}_{陆战队} &= 9.8 \times (1 + 0.3) = 12.74 \quad \text{(+3攻)} \\
\text{EHP}_{跳虫} &= \frac{35}{1 - \frac{0}{0 + 10}} = 35 \\
\text{CEV}_{陆战队 \rightarrow 跳虫} &= \frac{12.74}{35} = 0.364
\end{align}

反向计算:
\begin{align}
\text{DPS}_{跳虫} &= 7.2 \times (1 + 0.3) = 9.36 \\
\text{EHP}_{陆战队} &= \frac{45}{1 - \frac{0}{0 + 10}} = 45 \\
\text{CEV}_{跳虫 \rightarrow 陆战队} &= \frac{9.36}{45} = 0.208
\end{align}

根据兰彻斯特方程,1个陆战队约等于1.75个跳虫的战斗力。

\end{document}