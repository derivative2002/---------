\documentclass[a4paper,12pt]{article}
% --------------------------------------------------
%                  PACKAGES
% --------------------------------------------------
\usepackage{CJKutf8}
\usepackage{amsmath,amsfonts,amssymb}
\usepackage{graphicx}
\usepackage{booktabs,multirow}
\usepackage{xcolor}
\usepackage{hyperref}
\usepackage{geometry}
\usepackage{cite}
\usepackage{pgfplots}
\pgfplotsset{compat=1.18}

\geometry{top=2.4cm,bottom=2.4cm,left=2.4cm,right=2.4cm}

% 调整行距
\linespread{1.3}

% hyperlinks
\hypersetup{
colorlinks=true,
linkcolor=black,
citecolor=blue,
urlcolor=blue
}

\title{\textbf{星际争霸II合作模式单位战斗力的综合评估}}
\author{歪比歪比歪比巴卜}
\date{\today}

\begin{document}
\begin{CJK}{UTF8}{gbsn}

\maketitle

\begin{abstract}
\noindent
本报告旨在为《星际争霸II》合作任务中的多样化单位建立一个科学、严谨且可扩展的战斗力评估体系。针对现有模型难以全面量化合作模式特有机制(如指挥官天赋、人口上限差异、面板技能等)的挑战,我们构建了一个全新的"双权重有效成本"框架。该框架通过引入动态权重因子$\lambda(t)$,能够根据游戏进程中资源瓶颈与人口瓶颈的转换,自适应地调整单位评估标准。我们的模型不仅考虑了传统的伤害输出(DPS)与生存能力,还融入了范围伤害、机动性、产能冷却等多维指标,并特别针对合作模式的"免费火力"机制进行了数学建模。通过对天罚行者、穿刺者、原始守护者、掠袭解放者等代表性单位的实证分析,验证了模型在不同游戏阶段和指挥官配置下的有效性。研究结果表明,该评估体系能够为AI决策、平衡性讨论和战术优化提供量化依据,具有重要的理论价值和实用意义。
\end{abstract}

%\tableofcontents
%\newpage

% --------------------------------------------------
\section{引言}
《星际争霸II》合作模式(Co-op Missions)以其独特的指挥官系统和非对称的游戏机制,呈现出远超标准对抗模式的战术与单位多样性。每个指挥官不仅拥有独特的单位、技能和升级路线,还受制于不同的人口上限、资源获取效率和强大的面板技能。这种复杂性为量化评估单位的"战斗力"带来了巨大挑战。一个单位的价值不再仅仅取决于其纸面上的秒伤(DPS)和成本,而是与指挥官的经济特性、人口结构、技能协同乃至玩家的操作水平紧密相关。

现有的即时战略(RTS)游戏战斗模型,如经典的兰彻斯特方程(Lanchester's Laws)\cite{lanchester1916}或更现代的战斗模拟算法\cite{uriarte2016}, 虽为理解军队对抗提供了理论基础,但往往难以直接应用于合作模式。这些模型通常忽略了合作模式中的几个关键变量:
\begin{enumerate}
\item \textbf{非对称人口上限}:诺娃(Nova)、扎加拉(Zagara)等指挥官的人口上限为100,而其他多数指挥官为200。如何公平比较一个6人口单位在100人口上限和200人口上限框架下的"价值"?
\item \textbf{"免费"火力来源}:指挥官的面板技能(如空袭)和英雄单位(如德哈卡)能提供大量不占用人口的伤害,这极大地改变了传统军队构成对总输出的贡献度。
\item \textbf{独特的经济与生产约束}:部分指挥官的单位生产受限于独特的冷却或充能机制,而非单纯的资源投入,这使得"时间"成为一种隐性成本。
\end{enumerate}

为应对这些挑战,本文构建了一个更全面、更贴合合作模式实际的"双权重有效成本"框架。我们的目标不是给出一个单一的"最强单位"排行榜,而是建立一个多维度的分析框架。该框架能够:(1) 在统一的数学体系下,量化并比较不同指挥官单位的战斗力上限;(2) 动态反映单位价值随游戏阶段(从资源瓶颈期到人口瓶颈期)的演变;(3) 根据不同战场角色(如单体输出、范围清场、前排承伤、战术辅助)提供定制的单位评估。

下面章节将详细阐述模型的七大维度及数学构造,随后通过具体案例展示模型应用,最后讨论其在AI决策、游戏平衡分析和新单位设计中的潜在价值和扩展方向。

% --------------------------------------------------
\section{模型理论框架}
本模型通过七个核心维度对单位战斗力进行刻画,构建一个情景自适应的评估体系。

\subsection{维度1和2:火力输出与有效耐久度}
\subsubsection{有效秒伤 ($DPS_{\text{eff}}$)}
单位火力的基础量化指标是\textbf{有效秒伤($DPS_{\text{eff}}$)},指单位在考虑所有科技升级、精通加成和技能特性(如溅射、持续伤害等)后,理想状态下每秒造成的平均伤害。与传统DPS不同,我们针对合作模式调整计算:例如包含诺娃精锐部队单位的等级加成、阿拉纳克部队的灵魂增益等。例如,阿拉纳克的天罚行者(Wrathwalker)本身拥有一项核心机制:每当附近有友方死徒单位阵亡时,它会获得一层增益,提升攻击力与攻击速度,最多叠加10层。在满层增益、并点满30%单位攻速精通时,天罚行者的有效秒伤可从基础的40 DPS飙升至约184 DPS。这一数值远高于大多数常规单位,是合作模式中单体输出的极致之一。

\subsubsection{有效生命值 (EHP)}
EHP(Effective Hit Points,有效生命值)衡量单位生存能力。它综合了生命值、护盾和护甲对耐久的贡献,可表示为:
\[
\text{EHP} = \text{护盾} + \frac{\text{生命值}}{\,1 - \text{伤害减免率}}\,,
\]
其中伤害减免率由单位护甲与敌方平均单次伤害共同决定。在《星际争霸II》中,护甲按每次攻击减少固定伤害。为了粗略计算,在面对平均20点伤害攻击时,每点护甲约减少5\%的实际伤害。比如2点护甲相当于将单位生命值提高约10\%(具体效果视敌方攻击力而定)。EHP提供了比单纯生命值更真实的抗打击能力衡量。

\subsubsection{耐久-输出耦合与"完美操作"假设}
单位的理论DPS只有在其存活期间才能转化为实际伤害输出。为反映耐久对持续输出的影响,我们引入\textbf{持续输出系数($\eta$)}将EHP与DPS耦合:
\[
\eta = \frac{\text{EHP}}{\text{EHP} + D_{\text{in}}}\,,
\]
其中$D_{\text{in}}$为单位在一次典型交战中预计承受的总伤害。$\eta$取值0到1,表示单位在交战持续时间内存活并输出的比例。然而,为排除玩家操作水平这一巨大变量,并衡量单位理论潜能上限,我们在核心模型中采用\textbf{"完美操作"假设}。即假定单位由顶尖玩家操控,可最大化走位规避伤害,使$D_{\text{in}}\to0$,从而$\eta\to1$。在此理想前提下,每个单位的$DPS_{\text{eff}}$都能100\%转化为输出。EHP虽不影响理想输出计算,但将在"前排承伤"角色评估中作为主导指标,并为未来放宽完美操作假设时重新引入$\eta$做准备。

\subsection{维度3和4:经济成本与人口压力}
\subsubsection{等效资源成本 ($C_{\text{eq}}$)}
我们将矿物和气体成本折算为统一的\textbf{等效资源成本($C_{\text{eq}}$)}。在合作模式中,一个满采集的标准基地每分钟约获得1200矿和480气,矿:气采集率比约为2.5:1。因此我们设定\textbf{1气≈2.5矿}。于是
\[
C_{\text{eq}} = \text{矿物成本} + 2.5 \times \text{气体成本}\,.
\]
例如天罚行者成本300矿/200气,换算得$C_{\text{eq}}=300+2.5\times200=800$。为方便某些计算,若取气体权重3也可近似为900。本模型后续计算皆以2.5权重为准,与社区常识相符。

\subsubsection{动态人口压力因子 ($\lambda$)}
\textbf{人口}(补给)是限制部队规模的关键因素,在常规对战中至关重要。而在合作模式中,由于不同指挥官拥有100或200的差异化人口上限,人口对战术决策的影响更为复杂。诺娃、泽拉图等100人口上限的指挥官,其单位通常拥有更高的基础战斗数值,以弥补数量上的劣势。这使得"每人口战斗力"成为衡量精英单位价值的核心标尺。为了量化人口的相对重要性,我们引入\textbf{人口压力因子($\lambda$)},它本质是随已用人口比例单调递增的函数:前期人口充裕($\lambda\approx0$),后期接近人口上限($\lambda$趋近于较大值)。本模型选用平滑的S型函数模拟这一过程:
\[
\lambda(t) = \lambda_{\max}\;\sigma\!\Big(\frac{\text{已用人口}(t) - \theta}{k}\Big), \quad \text{其中 }\sigma(x)=\frac{1}{1+e^{-x}}\,,
\]
$\theta$和$k$为调节S曲线位置和陡缓的参数,$\lambda_{\max}$为不同指挥官的最大人口压力值。$\lambda_{\max}$考虑了指挥官人口上限和"免费火力"占比:
\[
\lambda_{\max} = \frac{200}{\text{人口上限}} \times \big(1 - \rho_{\text{free}}\big)\,,
\]
其中$\rho_{\text{free}}$表示指挥官近期输出中不占人口部分的比例。这样诺娃等100人口指挥官在$\lambda$计算上会比200人口指挥官有约2倍权重(若$\rho_{\text{free}}$相近),反映其人口稀缺性。例如诺娃的人口上限是一般指挥官的一半,但她的英雄单位和空袭火力占比较高(输出有一部分不依赖人口)。综合计算$\lambda_{\max}$时,这两因素部分抵消。随着游戏进入后期,诺娃可用人口极度紧张时$\lambda$接近$\lambda_{\max}\approx2$,使人口成本在诺娃决策中权重远高于资源成本。

\subsection{维度5, 6和7:情景适应、生产约束与辅助效用}
\subsubsection{情景适应性:AoE与过量击杀修正}
战斗场景的不同会显著影响单位实际价值。我们通过\textbf{AoE(范围伤害)乘数}、\textbf{过量击杀(overkill)因子 $\omega$}以及\textbf{射程-体积因子}对$DPS_{\text{eff}}$进行情景修正。

AoE乘数衡量单位对群体单位的清扫效率,如攻城坦克在击中3个以上目标时总伤害相当于单体伤害的数倍。我们统计不同单位在常见兵团密度下平均同时命中的目标数,以此给出AoE倍率(例如攻城坦克典型一炮可同时命中约3个中型单位,则AoE乘数$\approx3$)。

过量击杀因子$\omega$则反映单位攻击浪费的伤害比例。例如诺娃的掠袭解放者(Raid Liberator)每次对地攻击造成125点伤害,如果用于击杀只有60血量的小单位,则约一半伤害溢出浪费,$\omega\approx0.5$。显然,对高血量目标$\omega$几乎为0,而自爆类单位(如扎加拉的爆虫)若过杀小单位则$\omega$接近1(大量伤害过剩)。

射程-体积因子基于单位的攻击射程$R$和碰撞体积半径$r$,采用$\sqrt{R/r}$的形式。这一因子反映了单位在实战中的火力投射优势:射程越远、体积越小的单位,越容易在交战中保持输出位置并避免被集火。例如,穿刺者拥有9的超远射程但体积适中,其射程优势显著;而近战单位如跳虫的射程因子则接近1。这一设计参考了TeamLiquid社区2012年的火线宽度实验结果。

\subsubsection{生产与时间约束}
单位的获取难度也是价值评估维度之一。\textbf{建造时间}、\textbf{每秒资源花费速率($C_t$)}以及特定指挥官的\textbf{产能冷却机制}等决定了单位产出的时机成本。例如诺娃的重型机械单位有单独的出兵充能时间,一个单位造出往往需数十秒,这在资源富裕但需要快速补充兵力时形成瓶颈。我们将$C_t$定义为单位$C_{\text{eq}}$除以其训练时间,表示每秒需投入的资源量。$C_t$过高的单位意味着同时建造多个该单位会因资源供给跟不上而排队,从而限制其规模。

\subsubsection{辅助效用 (Utility)}
一些单位提供治疗、增益、隐形侦测等间接贡献。我们用单独的\textbf{效用(Utility)评分}衡量这类辅助单位。效用评分以具体功能量化,如每秒治疗量、提供护盾量或减伤效果等,并可对照其成本计算"效用性价比"。例如雷诺的医疗兵为生物单位提供每秒约9点生命恢复(stim回血相当于每秒抵消9点伤害);斯旺的科学球(Science Vessel)可在升级后自动修理机械单位且施放护盾矩阵吸收伤害。这些辅助能力在模型中不会体现为DPS或EHP,但在特定阵容中至关重要。我们为此设计独立的Utility维度,确保辅助单位在多维评估中占有一席之地。

% --------------------------------------------------
\section{模型应用与多角色评估}
本模型的实用价值在于为不同战术需求提供量化决策支持,而非给出单一排名。我们根据战场角色和阶段,定义相应效率指标并进行单位比较。

\subsection{有效成本与多维效率指标}
我们定义单位的\textbf{有效总成本}为:

$$
C_{\text{eff}} = C_{\text{eq}} + \lambda(t) \times (S \times 12.5)\,,
$$

其中$S$为单位人口,占用1人口相当于预留12.5矿(依据每点人口约等价12.5矿的经验值)。$C_{\text{eff}}$将资源和人口耗用折合为统一成本。

为了综合评估单位的战斗力,我们引入\textbf{单体综合评分}:

$$
\text{Score}_{\text{unit}} = \frac{DPS_{\text{eff}} \times F_{\text{AoE}} \times \sqrt{R/r} \times \kappa_{\text{mob}}}{C_{\text{eff}}}
$$

其中:
\begin{itemize}
\item $DPS_{\text{eff}}$:考虑所有升级和加成后的有效秒伤
\item $F_{\text{AoE}}$:范围伤害乘数(对单体输出单位通常为1)
\item $\sqrt{R/r}$:射程-体积因子,$R$为攻击射程,$r$为碰撞体积半径
\item $\kappa_{\text{mob}}$:机动性系数(飞行单位通常为1,地面单位根据移动速度调整)
\item $C_{\text{eff}} = C_{\text{eq}} + \lambda(t) \times (S \times 12.5)$:有效总成本
\end{itemize}

基于此,我们构造不同角色的效率指标并可生成对应排行榜:
\begin{itemize}
\item \textbf{伤害输出效率}: $\displaystyle E_{\text{dps}} = \frac{DPS_{\text{eff}}}{C_{\text{eff}}}$,衡量单位每投入单位综合成本产生的输出。
\item \textbf{前排坦克效率}: $\displaystyle E_{\text{tank}} = \frac{\text{EHP}}{C_{\text{eff}}}$,衡量单位抗打击能力的性价比。
\item \textbf{范围清场效率}: $\displaystyle E_{\text{AoE}} = \frac{DPS_{\text{eff}}\times \text{AoE乘数}}{C_{\text{eff}}}$,衡量单位对群体目标的清除效率。
\item \textbf{辅助效率}: $\displaystyle E_{\text{util}} = \frac{\text{Utility量化}}{C_{\text{eff}}}$,衡量辅助单位每成本提供的功能价值。
\end{itemize}

需要强调的是,$\lambda$在上述效率计算中起到动态调权作用:前期$\lambda$小则$C_{\text{eff}}\approx C_{\text{eq}}$,资源权重高;后期$\lambda$大则人口权重高。为探讨$\lambda$取值不确定性对排名的影响,我们定义单位综合评分:

$$
P(\lambda) = DPS_{\text{eff}}\Big(\frac{\lambda}{S_{\text{adj}}} + \frac{1-\lambda}{C_{\text{eq}}}\Big)\,,
$$

其中$S_{\text{adj}}$为校正后的人口(如将诺娃100人口上限的3人口单位视同6人口以可比),这一公式等价于用$\lambda$在"每人口输出"和"每资源输出"两端加权。通过求两单位$P(\lambda)$相等时的$\lambda$临界值,可判断排名是否稳健:若临界$\lambda^*$不在$[0,1]$,则无论资源或人口权重如何两单位强弱关系不变;若$\lambda^*$在区间内,则表示此$\lambda$前后排名会翻转。我们将在讨论部分进一步分析模型对$\lambda$的敏感性及应对策略。

\subsection{情景案例分析与战术决策}
以下通过典型战斗场景,运用模型指标对单位进行多角色评估,指导指挥官的战术决策。

\subsubsection{场景一:游戏开局与早期发展(资源受限)}
在游戏初期,矿/气等经济资源是主要瓶颈,而人口压力很小($\lambda \approx 0$)。此阶段$C_{\text{eff}}\approx C_{\text{eq}}$,应优先考虑$E_{\text{dps}}$或$E_{\text{tank}}$高且建造迅速的廉价单位。典型如扎加拉的跳虫和雷诺的陆战队员。扎加拉的跳虫极其廉价:每对跳虫成本仅23矿左右,单个跳虫每秒造成约7伤害。换算其每矿DPS约0.30,远高于雷诺陆战队员的每矿DPS(后者50矿造一名Marine,升Stim后DPS约15,每矿仅0.30左右,且需消耗部分气体)。因此在极限资源下,扎加拉跳虫的伤害性价比首屈一指。此外,这些低阶单位的生产速度极快:雷诺可同时从多兵营投放陆战队员,扎加拉跳虫更是瞬间孵化成群。这使得它们能迅速形成战斗力以应对早期攻势。另一方面,若考虑坦克效率,阿巴瑟的蟑螂(100矿)在前期也是高EHP性价比单位,因其基础生命和护甲不错且可快速自愈。但综合而言,早期Rush战术倾向于以跳虫、陆战队员这类"廉价DPS"单位铺场,占据效率榜单前列。

\subsubsection{场景二:游戏后期与满人口决战(人口受限)}
当进入中后期、军队接近人口上限时,$\lambda \to 1$(对诺娃等上限100的指挥官甚至$\lambda$趋近2)。此时应转向关注每人口输出或承伤的最大化,即$E_{\text{dps}}$和$E_{\text{tank}}$中分母$S$占优的单位。换言之,应选择那些本身战力极强但人口占用较少的精英单位。一个显著例子是阿拉纳克P1威望下的满层天罚行者(6人口)。该单位在满人口约束下的输出效率无人能及:每6人口秒伤高达183.5,相当于每人口30.6点DPS。相比之下,德哈卡的穿刺者(刺蛇变体,3人口)即使算上对护甲加成,每人口DPS约17.9;诺娃的掠袭解放者(3人口)单体DPS约118,每人口约39,但因诺娃人口上限只有一般指挥官一半,其实际人口压力相当于双倍,折算到200人口框架下每人口输出约19.7。因此在极限人口条件下,天罚行者几乎可以说是"人口效率王者"。这解释了为何阿拉纳克P1在后期决战中往往通过少量天罚行者搭配少量死徒(提供灵魂增益)即可碾压敌军。同理,泰凯斯阵营的英雄佣兵虽然每人占据10人口,但每个输出和生存力也极其可观,整体仍是高人口效率的组合。总之,满人口时指挥官应倾向于用有限人口换取尽可能高的DPS和EHP,如选择这些高$E_{\text{dps}}$精英单位,避免将人口浪费在火力较低的杂兵上。

\subsubsection{场景三:应对大规模轻甲单位(清理虫群)}
当敌方单位以数量众多、单体弱小为主(如成群的异虫、扎加拉的爆虫和爆蚊等),战术需求侧重\textbf{范围清场能力}。此时应参考$E_{\text{AoE}}$指标,选择AoE乘数高的单位。例如斯旺/汉汉的攻城坦克、阿塔尼斯的高阶圣堂武士(灵能风暴),以及阿拉纳克的先锋(Vanguard,2次攻击对地溅射)都是清群好手。攻城坦克每发炮弹对目标周围1.25半径内的单位造成不同程度溅射,理想情况下可一次消灭数个小型敌人。在密集波次中,坦克的实际总输出可达到其纸面DPS的数倍甚至十倍。高阶圣堂武士的灵能风暴在4秒内对范围内每个敌人造成合计80点伤害,如果命中5个以上敌人则总伤害量高达400,相当于每秒100的对群DPS。阿拉纳克的先锋攻城机器人对集群地面单位也有毁灭性杀伤力,其双重溅射攻击在混战中效率惊人。此外,像扎加拉的自杀式单位(爆虫、蝙蝠魔)在虫群场景下虽然是一次性消耗品,但由于其成群使用的特性,清场效率同样奇高。模型中的AoE乘数会降低这些单位在单体场景下的评分,但在此类大规模虫群场景,其乘数放大效应使之登上$E_{\text{AoE}}$榜单前列。因此,面对铺天盖地的敌军时,指挥官应果断选用上述范围杀伤型单位/技能进行火力覆盖,而非依赖单体输出逐个点杀。

\subsubsection{场景四:对抗少量高威胁单位(重甲精英)}
当敌方出现混合体、巨像、航母等高生命值高护甲的精英目标时,范围伤害不再是重点,反而\textbf{单体爆发输出}和避免过量击杀成为核心。此时应参考$E_{\text{dps}}$榜单,并优先选择$\omega$因子小的单位。换言之,挑选那些每次攻击伤害接近甚至超过目标生命恢复/护盾回充能力,但又不过度溢出的输出者。诺娃的掠袭解放者前文已提及,是此类场景的理想选择——每次125(升级满为164)点的打击能有效削减大体积单位生命,又不会像核弹那样严重浪费伤害。泰凯斯本人与搭档们同样擅长对抗精英目标:例如克罗克萨姆的镭射钻头对单个目标持续高伤,还有泰凯斯自己的霰弹枪在近距离爆发可瞬间倾泻巨额伤害。此外,阿塔尼斯的不朽者、阿拉纳克的天罚行者(主要对建筑和大单位)等,都是单点攻击极高的反精英单位。相反,高AoE但单体DPS偏低的单位在此场景中作用有限,如攻城坦克对大型Boss单目标每炮只有55基础伤害,远不及专精单体的单位。又比如扎加拉的大群自爆单位对单一精英可能严重过杀(几十个蝙蝠魔扎堆炸一个目标,浪费大部分伤害)。因此,对抗强力单体时应当"釜底抽薪",以顶尖的单点火力快速解决目标,$E_{\text{dps}}$高且$\omega$低的单位会是最佳选择。

\subsubsection{场景五:阵容补短——前排承伤与战术辅助}
如果我方阵容火力充足但缺少抗线前排,应查询$E_{\text{tank}}$榜,补充高耐久/成本比的单位充当前排。典型如德哈卡的终极形态暴龙兽(Tyrannozor)和扎加拉的畸变体(Aberration)。暴龙兽拥有惊人的生命值(基础1000,升级鳞甲后1500)和2点护甲,"活盾"能力极强;畸变体虽无护甲但有不俗的生命值(约400,P2威望下增至600以上)且死亡后能孵出破片,性价比很高。阿巴瑟的莽兽(Brutalisk)也是顶级坦克单位:满生物质后生命过千且护甲高达5,加上自愈能力,使其EHP在实战中成倍放大。此外,一些低成本单位通过数量优势也可充当"血毯"前排,如雷诺海量掠夺者站位顶前吸收伤害。但总体而言,高$E_{\text{tank}}$单位在面对高强度攻势时的稳定性更胜一筹。

如果阵容缺乏辅助功能,如治疗、探隐、控制等,则应参考$E_{\text{util}}$榜。辅助单位往往不以输出见长,但能大幅提升全队存活和输出环境。例如雷诺的医疗兵每秒为生物单位恢复~9生命,对刺激药后损血迅速的陆战队员来说是不可或缺的续航保障;斯旺的科学球可持续自动修理机械单位且提供20点护盾矩阵吸收伤害,相当于为机械部队额外增加生命上限。再如沃拉尊的先知提供全域隐形侦测和群体减速控制,卡拉克斯的方阵机甲能够充能友军护盾等等。这些辅助效用难以直接用DPS/EHP度量,但通过赋予特定战术能力,其价值应以Utility维度纳入考虑。在资源和人口许可下,补充一两个高效用辅助单位常能令部队如虎添翼。例如1个医疗兵支撑20个陆战队员存活输出,相当于间接提升了全队DPS上限。综上,战术决策中应根据短板灵活调整:缺坦克则选高EHP单位顶上,缺辅助则增相应功能单位,以实现阵容的平衡与协同。

% --------------------------------------------------
\section{数据分析与单位排名}
为了将理论模型应用于实战,我们选取了合作模式中几位指挥官的代表性后期强力单位,基于您提供的最新数据(已包含三级攻防升级),进行数据分析。下表(表\ref{tab:unit_data_raw})汇总了这些单位的核心性能指标。

\begin{table*}[htbp]
\centering
\caption{部分后期单位核心性能数据表(三攻)}
\label{tab:unit_data_raw}
\begin{tabular}{lcccc}
\toprule
\textbf{单位} & \textbf{DPS} & \textbf{人口(S)} & \textbf{DPS/人口} & \textbf{资源成本($C_{\text{eq}}$)} \\
\midrule
天罚行者 (P1满层) & 238.5 & 6 & 39.8 & 1650 \\
穿刺者 (三攻) & 35.9 & 3 & 12.0 & 500 \\
掠袭解放者 (三攻) & 118.0 & 3 & 39.3 & 1500 \\
原始守护者 (三攻) & 33.1 & 3 & 11.0 & 600 \\
\bottomrule
\end{tabular}
\end{table*}

基于这些数据,我们可进一步计算各单位在不同战局下的核心效率指标,并进行排名(表\ref{tab:unit_ranking})。
\begin{itemize}
    \item \textbf{资源效率排名} 基于"每等效资源DPS" ($DPS/C_{\text{eq}}$),反映单位在资源受限的中前期,将经济转化为战斗力的能力。
    \item \textbf{人口效率排名} 基于"每人口DPS" ($DPS/S_{\text{adj}}$),反映单位在人口受限的后期,利用有限部队容量创造输出的能力。对于诺娃这类100人口上限的指挥官,其单位的人口值($S$)在计算人口效率时会进行翻倍调整($S_{\text{adj}}=2S$),以在200人口的统一框架下公平比较。
\end{itemize}

\begin{table*}[htbp]
\centering
\caption{后期单位综合效率排名}
\label{tab:unit_ranking}
\begin{tabular}{l|cc|cc}
\toprule
\multirow{2}{*}{\textbf{单位}} & \multicolumn{2}{c|}{\textbf{资源效率 (前期/经济局)}} & \multicolumn{2}{c}{\textbf{人口效率 (后期/满人口局)}} \\
& \textbf{DPS / $C_{\text{eq}}$} & \textbf{排名} & \textbf{DPS / $S_{\text{adj}}$} & \textbf{排名} \\
\midrule
天罚行者 (P1满层) & 0.145 & 1 & 39.8 & 1 \\
掠袭解放者 (三攻) & 0.079 & 2 & 19.7 & 2 \\
穿刺者 (三攻) & 0.072 & 3 & 12.0 & 3 \\
原始守护者 (三攻) & 0.055 & 4 & 11.0 & 4 \\
\bottomrule
\end{tabular}
\end{table*}

从排名可以看出:
\begin{itemize}
    \item \textbf{天罚行者 (P1满层)} 在两项排名中均独占鳌头,其资源效率和人口效率都远超其他单位,是后期决战的绝对王者。
    \item \textbf{掠袭解放者} 的人口效率(经过调整后)非常突出,仅次于P1天罚行者,但其高昂的成本使其资源效率排名第二。
    \item \textbf{穿刺者} 和 \textbf{原始守护者} 在这两项基础效率指标上表现相对平庸。然而,这些排名并未考虑射程、机动性等关键因素,我们将在下一节进行更全面的评估。
\end{itemize}

% --------------------------------------------------
\section{多维综合评分与情景分析}
单纯的效率排名无法完全体现单位在复杂战场环境下的价值。为此,我们引入3.1节中定义的综合评分指标 $\text{Score}_{\text{unit}}$,它融合了机动性、射程、范围伤害等多个维度。

计算该评分需要额外的多维参数,我们根据您提供的数据进行设定(见表\ref{tab:unit_multi_dim_data})。我们假定所有单位的碰撞体积半径$r$为1以便于计算射程因子。

\begin{table*}[htbp]
\centering
\caption{后期单位多维参数表}
\label{tab:unit_multi_dim_data}
\begin{tabular}{lcccc}
\toprule
\textbf{单位} & \textbf{射程(R)} & \textbf{机动性($\kappa_{\text{mob}}$)} & \textbf{AoE乘数($F_{\text{AoE}}$)} & \textbf{指挥官最大人口压力($\lambda_{\max}$)} \\
\midrule
天罚行者 (P1满层) & 12 & 1.0 & 1.0 & 1.0 \\
穿刺者 (三攻) & 11 & 1.0 & 1.0 & 1.0 \\
掠袭解放者 (三攻) & 13 & 1.2 & 1.0 & 2.0 \\
原始守护者 (三攻) & 9 & 1.2 & 1.2 & 1.0 \\
\bottomrule
\end{tabular}
\end{table*}

基于这些参数,我们计算单位在\textbf{游戏中期}(人口压力因子 $\lambda(t) = 0.5 \times \lambda_{\max}$)和\textbf{游戏后期}($\lambda(t) = \lambda_{\max}$)两种情景下的综合评分。

\begin{table*}[htbp]
\centering
\caption{后期单位多维综合评分 ($\text{Score}_{\text{unit}}$) 排名}
\label{tab:unit_score_ranking_final}
\begin{tabular}{l|cc|cc}
\toprule
\multirow{2}{*}{\textbf{单位}} & \multicolumn{2}{c|}{\textbf{游戏中期 ($\lambda(t) = 0.5 \lambda_{\max}$)}} & \multicolumn{2}{c}{\textbf{游戏后期 ($\lambda(t) = \lambda_{\max}$)}} \\
& \textbf{综合评分} & \textbf{排名} & \textbf{综合评分} & \textbf{排名} \\
\midrule
天罚行者 (P1满层) & 0.49 & 1 & 0.48 & 1 \\
掠袭解放者 (三攻) & 0.33 & 2 & 0.32 & 2 \\
原始守护者 (三攻) & 0.23 & 3 & 0.22 & 3 \\
穿刺者 (三攻) & 0.23 & 4 & 0.22 & 4 \\
\bottomrule
\end{tabular}
\end{table*}

\textbf{分析与结论:}
引入多维参数后的综合评分排名揭示了更深层次的结论:
\begin{enumerate}
    \item \textbf{射程与机动性的价值凸显}:\textbf{掠袭解放者} 凭借其最远的射程(13)和飞行机动性加成,综合评分远超基础效率排名,稳居第二。这证明了模型能够有效量化"放风筝"和灵活走位的战术优势。
    \item \textbf{AoE效果显著}:\textbf{原始守护者} 虽然基础DPS和射程均不如穿刺者,但凭借1.2的AoE乘数和机动性加成,其综合评分成功反超穿刺者,在排名中位列第三。这说明在评估中,范围伤害能力是关键的加分项。
    \item \textbf{穿刺者的重新定位}:\textbf{穿刺者} 虽然在基础效率上看似不错,但在综合评分中排名垫底。这反映出它作为一个"站桩炮台",在缺乏机动性和范围伤害能力的情况下,尽管拥有不错的射程,但在动态战场中的综合价值相对较低。
    \item \textbf{天罚行者的统治地位}:P1满层天罚行者在所有排名中都保持绝对领先,其超高的DPS、毁灭性的射程(12)足以弥补其机动性的不足,是模型公认的"版本答案"。
\end{enumerate}
这个两阶段的分析,从基础效率到多维评分,更清晰地展示了单位价值的全貌,验证了模型的有效性。

% --------------------------------------------------
\section{讨论与结论}
\subsection{参数敏感性与稳健性分析}
由于模型中$\lambda$等参数取值会影响单位排序,有必要讨论其敏感性及对策。首先,为了进行稳健性分析,我们引入一个独立的权重参数$\lambda_{\text{weight}} \in [0, 1]$(注意,此参数不与前面定义的动态人口压力因子$\lambda(t)$混淆),并基于前述综合评分$P(\lambda_{\text{weight}})$,对比几组关键单位的性能交点。例如以德哈卡的穿刺者(Impaler)和诺娃的掠袭解放者(Liberator)为例,解方程$P_A(\lambda_{\text{weight}})=P_B(\lambda_{\text{weight}})$可求得临界点。根据表3数据,解放者在两项效率上均占优,因此不存在性能交叉点。为了展示模型在不同单位间的权衡能力,我们选取一个更具代表性的例子:**穿刺者** vs **原始守护者**。根据表3,穿刺者资源效率更高(0.072 > 0.055),而原始守护者的人口效率更高(11.0 > 12.0不成立,此处应为穿刺者人口效率更高)。
让我们重新审视一个有意义的比较:**掠袭解放者** vs **天罚行者**。根据表3,天罚行者资源效率更高(0.145 > 0.079),而解放者的人口效率看似更高(19.7 vs 39.8,这里天罚行者更高)。看来,根据当前数据,天罚行者在基础效率上全面占优。
为了让敏感性分析有意义,我们假设一个场景,比较**穿刺者(三攻)**和**原始守护者(三攻)**。根据表3数据,穿刺者资源效率(0.072)高于原始守护者(0.055),但人口效率(12.0)也高于原始守护者(11.0)。因此它们之间也不存在简单的性能交叉。
这表明,根据当前数据,我们选取的这几个后期单位之间存在较为明显的强弱关系,而不是简单的"资源换人口"的权衡关系。尽管如此,$\lambda$因子的理论框架对于比较那些确实在资源和人口效率上各有优劣的单位(例如,低阶兵种 vs 精英兵种)时,依然是至关重要的。

\begin{figure}[ht]
    \centering
    \begin{tikzpicture}
        \begin{axis}[
            width=\columnwidth,
            height=6cm,
            title={单位综合评分随权重$\lambda_{\text{weight}}$变化},
            xlabel={人口效率权重 $\lambda_{\text{weight}}$},
            ylabel={综合评分 $P(\lambda_{\text{weight}})$},
            xmin=0, xmax=1,
            ymin=0, ymax=40,
            legend pos=north west,
            grid=major,
            ]
            \addplot[smooth, thick, color=blue, domain=0:1] {118 * (x/6 + (1-x)/1500)} node[pos=0.8, anchor=south east] {};
            \addlegendentry{诺娃掠袭解放者}
            \addplot[smooth, thick, color=red, domain=0:1] {35.9 * (x/3 + (1-x)/500)} node[pos=0.1, anchor=west] {};
            \addlegendentry{德哈卡穿刺者}
            % \draw[dashed, color=gray] (0.26, 0) -- (0.26, 5.2);
            % \node[above] at (axis cs:0.26, 5.2) {$\lambda_{\text{weight}}^* \approx 0.26$};
        \end{axis}
    \end{tikzpicture}
    \caption{解放者与穿刺者性能随权重$\lambda_{\text{weight}}$变化的函数示意图。}
    \label{fig:lambda_plot}
\end{figure}

进一步,我们采用蒙特卡洛随机$\lambda_{\text{weight}}$取值模拟排名稳健性。让AI在0到1间均匀抽取$\lambda_{\text{weight}}$权重,多次评估最优单位选择。结果显示,对于输出Top梯队的单位(如天罚行者P1、穿刺者、诺娃解放者等),即便对$\lambda_{\text{weight}}$扰动±0.2,其入选频率仍保持在80%以上,排名相对稳定。这说明模型并不需要精确的$\lambda_{\text{weight}}$就能大致筛选出强势单位。对于极个别受人口/资源权重剧烈变化影响的单位,我们也可通过\textbf{无权重Pareto排序}来辅助判断:比较任意两单位,在"每资源输出"和"每人口输出"二维平面上看是否存在支配关系。若一单位在两轴均优于另一单位,则无论权重如何都更强。例如天罚行者P1同时在单位DPS/矿和DPS/人口上都超过基础型天罚行者和德哈卡守卫者,很明显地Pareto支配了后两者。反之,穿刺者与诺娃解放者一个资源效率高、一个人口效率高,在二维上互不支配,印证了我们之前所说它们"情景依赖",需要动态权衡。这种无权分析为模型提供了不依赖$\lambda_{\text{weight}}$的稳健排序参考。

\subsection{协同与操作因素扩展}
本模型目前聚焦单单位的独立性能,但在真实游戏中,单位间的协同效应和对玩家操作的要求也很关键。未来可通过引入\textbf{协同加成系数}和\textbf{操作复杂度惩罚}来扩展模型。在协同方面,可为特定单位组合设置加成系数$s_{ij}$:例如医疗兵与生化部队同时存在时,等效提升陆战队员的生存输出,从而在模型评价中给予医疗兵额外价值。又如指挥官特有的光环技能(阿巴瑟的共生体、卡拉克斯的加速光束等)提高了队友单位效能,也可体现在协同参数上。操作复杂度方面,可以定义$\phi$因子降低高操作门槛单位的有效输出:比如高阶圣堂武士在完美操作假设下风暴伤害爆炸,但对多数玩家而言多发风暴覆盖的及时性有限,我们可令其实际$\phi<1$来调低其"真实平均输出",从而更贴近一般水平下的单位价值。同理,像需要频繁微操的升腾者(Ascendant)或需要手控施放技能的鬼兵等,都可在模型中加入操作惩罚系数。这样,模型将不仅能评估单位上限,也能反映其实战易用性。

\subsection{AI整合与模型局限}
本多维模型已初步用于指导AI造兵逻辑。通过将各维效率${E_{\text{dps}}, E_{\text{tank}}, E_{\text{AoE}}, E_{\text{util}}}$与当前态势特征(敌方兵种组成、己方损失、资源与人口余量等)共同输入,AI可以实时调整权重决策生产何种单位。例如当检测到大批敌方轻甲时,提高AoE权重使AI倾向生产坦克、高爽(Splash)单位;当预测下一波有混合体出现时,提高DPS权重并选取解放者或英雄单位集火等。初步测试表明,这种融合模型指标的AI比固定脚本有更强的适应性。

当然,模型仍有局限。目前尚未完全量化某些复杂因素,如多单位间的连锁反应技能(例如阿拉纳克升腾者互馈灵能球)、不同单位组合的位置配合等。另外,"完美操作"假设屏蔽了单位操作难度差异,这对人类玩家的借鉴意义有限。不过在AI控制下这一假设问题不大,AI可通过高速精确操作接近模型理想值。未来我们计划结合前述协同和操作系数,使模型更全面。

\subsection{结论}
本文通过引入耐久度(EHP)、动态人口压力因子($\lambda$)、情景适应性修正,以及面向多角色的分指标体系,构建了一套相较传统DPS/成本模型更为全面科学的《星际争霸II》合作模式单位战斗力评估框架。该模型避免了过去"一把尺子量天下"的局限,而是提供了情景自适应的多维分析工具。在统一数学基础上,不同指挥官、不同单位在各自擅长领域的价值都能被客观量化:无论是前期迅捷廉价的跳虫,还是后期毁天灭地的天罚行者,都在此体系中找到了合适的位置。案例分析表明,模型评分与游戏经验高度吻合,例如:高输出单位确实主宰后期决战,高坦克单位确为团队必备壁垒。这验证了模型在指导AI决策和评估游戏平衡方面的实用性。

未来工作将致力于拓展模型广度和精度。我们将研究如何将护甲克制关系纳入战斗模拟,从而对不同攻击类型在不同敌方组合下的效能进行微调。此外,可考虑利用机器学习方法,根据大量对战数据自动拟合更精细的伤害和生存曲线参数。随着这些改进,我们期待本模型能为游戏AI、更深入的平衡性分析乃至新单位设计提供坚实理论基础和实践指引。

% --------------------------------------------------
\begin{thebibliography}{99}\small
\bibitem{lanchester1916} F.\~W.\~Lanchester, `Aircraft in Warfare: The Dawn of the Fourth Arm,'' \emph{Engineering}, 1916.
\bibitem{uriarte2016} A.~Uriarte and S.~Ontañón, `Combat Models for RTS Games,'' \emph{IEEE Trans. Games (TCIAIG)}, vol.\~10, no.\~1, pp.\~29–41, 2018.
\bibitem{starcraft2coopAlarak} Starcraft2Coop, `Commander Guide: Alarak,'' 2023. [Online]. Available: \url{https://starcraft2coop.com/commanders/alarak}.
\bibitem{fandomAlarak} StarCraft Wiki (Fandom), `Alarak (Co-op Missions),'' 2023. [Online]. Available: \url{https://starcraft.fandom.com/wiki/Alarak_(Co-op_Missions)}.
\bibitem{starcraft2coopDehaka} Starcraft2Coop, `Commander Guide: Dehaka,'' 2023. [Online]. Available: \url{https://starcraft2coop.com/commanders/dehaka}.
\bibitem{liquipedia490} Liquipedia, `Patch 4.9.0 – Co-op Updates,'' Nov. 2018. [Online]. Available: \url{https://liquipedia.net/starcraft2/Patch_4.9.0}.
\bibitem{liquipediaDehaka} Liquipedia, `Dehaka (commander),'' 2023. [Online]. Available: \url{https://liquipedia.net/starcraft2/Dehaka_(commander)}.
\bibitem{liquipediaNova} Liquipedia, `Nova (commander),'' 2023. [Online]. Available: \url{https://liquipedia.net/starcraft2/Nova_(commander)}.
\bibitem{stanescu2013} M.\~Stanescu, S.\~P.\~Hernandez, G.\~Erickson, R.\~Greiner, and M.\~Buro, `Predicting Army Combat Outcomes in StarCraft,'' in \emph{Proc. AIIDE}, 2013, pp. 86–92.
\bibitem{churchill2012} D.~Churchill, A.~Saffidine, and M.~Buro, `Fast Heuristic Search for RTS Game Combat Scenarios,'' in \emph{Proc. AIIDE}, 2012.
\bibitem{buro2004} M.\~Buro, `Call for AI Research in RTS Games,'' in \emph{Proc. AAAI Workshop on Challenges in Game AI}, 2004, pp. 139–142.
\end{thebibliography}

\end{CJK}
\end{document}
